\chapter{Introduction\label{ch.intro}}%
\index{Introduction}

ABCml stands for Analysis of Bone Counts by Maximum Likelihood.  The term
refers to three different things:
\begin{enumerate}
\item A statistical method that was introduced by \citet{Rogers:JAS-27-111}
  and has been used in several subsequent publications
  \citep{Rogers:JAS-27-635,Rogers:AA-00-x,Rogers:JAS-00}.
\item A computer program that implements this method.
\item A package of computer program that includes ABCml--the--program along
  with several other programs that play supporting roles.
\end{enumerate}
This document describes ABCml--the--package, and includes a discription of
ABCml--the--program.  It also includes a brief description of
ABCml--the--statistical method, but you will have to look elsewhere for a full
description of the method \citep{Rogers:JAS-27-111}.

This introduction describes the goals and limitations of ABCml in general
terms.  Later chapters will fill in the details.

\section{Goals}

Animals are introduced into archaeological assemblages in various ways, each
of which will be referred to as an ``agent of deposition.''  These might
include natural deaths at the site, kills at the site by humans or non-human
carnivores, transport to the site by hunters or scavengers of various species,
and running water, and so on.  The primary goal of ABCml is to estimate the
relative contributions of different agents of deposition.

In order to achieve this goal, it is necessary also to estimate two additional
parameters, which account for factors that also influence bone counts.  The
first of these factors is the number of carcasses originally deposited, and
the second is the severity of destructive processes such as gnawing by
carnivores. 

ABCml provides both point estimates of these parameters and also standard
errors. 

\section{Data requirements}

ABCml requires, as input data, detailed information about the agents of
deposition that are to be studied.  The literature contains some data on bone
transport by human and non-human hunters, but our knowledge of this issue is
nonetheless woefully inadequate.  In the absence of extensive data on bone
transport, how can ABCml be used?

First, it is important to realize that this problem affects not only ABCml,
but also all other methods of analysis.  ABCml differs from other methods only
in that it requires that we make our assumptions explicit and detailed.  

Second, there is no special difficulty in incorporating conventional ideas
about transport and attrition into ABCml.  For example, there is a long
history of analyses that are based on the idea that hunters will first discard
those skeletal parts with greatest weight and lowest utility
\cite{White:AA-19-254,Perkins:SA-219-97}.  \citet{Rogers:JAS-00} show how this
qualitative idea can be incorporated into an analysis using ABCml.  The
results, of course, are no better than the assumptions, but that is true of
any method.  ABCml makes better use of its input data and violates fewer
assumptions than any alternative method of analysis that is currently
available. (For a critique of alternative methods, see
\citet{Rogers:JAS-27-111}, \citet{Rogers:JAS-27-635}, \citet{Rogers:AA-00-x},
and \citet{Rogers:JAS-00}.)

Third, there is a considerable body of empirical research regarding 
carcass transport by humans and other predators 
\citep{Binford:NE-78,Bunn:JAA-7-412,O'Connell:JAR-44-113,Marean:JAS-19-101}.
It is easy to incorporate the resulting data into analyses using ABCml.

In summary, ABCml makes optimal use of available data from all sources while
forcing us to be explicit about our assumptions.

\section{The programs}

The ABCml package consists of the following programs:
\begin{description}
\item[ABCml] estimates parameters and their standard errors, provides analysis
  of residuals.  If the input file contains a large number of data sets, ABCml
  will also provide quantiles of each estimated parameter.

\item[ABCsim] generates bone-count data by computer simulation.

\item[gnaw] subjects a data set to density-biased attrition.

\item[tabcfg] tabulates the configurations in a 
\hyperref{.cfg file}{.cfg file (see Section~}{)}{sec.cfg}, eliminating
  duplicates. 
  
\item[mau2cfg] converts data from .mau format (which is common in published
  literature) into \hyperref{.cfg format}{.cfg format}{}{sec.cfg} (which is
  required by several of the programs in this package).

\item[cplcfg] find the complement of a .cfg file.  If a .cfg file describes
  bones that were transported from kill site to home-base, then its complement
  will describe the bones that were left in the field.
\end{description}

%%% Local Variables: 
%%% mode: latex
%%% TeX-master: t
%%% End: 
