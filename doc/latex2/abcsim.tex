% LaTeX document produced by pod2latex from "abcsim.pod".
% The followings need be defined in the preamble of this document:
%\def\C++{{\rm C\kern-.05em\raise.3ex\hbox{\footnotesize ++}}}
%\def\underscore{\leavevmode\kern.04em\vbox{\hrule width 0.4em height 0.3pt}}
%\setlength{\parindent}{0pt}

\section{ABCSIM}%
\index{ABCSIM}

\subsection*{ABCSIM Documentation File}%
\index{ABCSIM Documentation File}

Abcsim is a program that generates one or more simulated archeological
data sets using data from input files and from the command line.  It
requires the following input files.

\begin{enumerate}

\item
A bone definition file, whose name ends with .bdf.  This file
describes the characteristics of the skeletal parts to be analyzed.

\item
Two or more agent definition files, each having a name ending with
.cfg.  Each of these files describes an agent of deposition.

\end{enumerate}

The format of these files is described in {\em files\/}.

In addition, the program recognizes the following command line
options:

\begin{description}

\item[{\tt -a x,x,...,x}]%
\index{-a x,x,...,x@{\tt -a x,x,...,x}}%
\hfil\\
Sets the vector of alpha values to the comma-separated string of
numbers given as an argument.  alpha{\tt [}i{\tt ]} is the fraction of the
assemblage attributable to the i'th agent of deposition.  By default,
all the entries of alpha have the same value.

\item[{\tt -b x}]%
\index{-b x@{\tt -b x}}%
\hfil\\
Set beta, the attrition parameter.  This determines how strongly the
simulated assemblages will be affected by attrition.  When {\tt beta=0},
no bones are lost to attrition.  When {\tt beta=1}, half the bones in a
complete skeleton would be lost.  By default, {\tt beta=0} so that no
bones are lost.

\item[{\tt -K x}]%
\index{-K x@{\tt -K x}}%
\hfil\\
Set number of animals in assemblage. Def: 100

\item[{\tt -n x}]%
\index{-n x@{\tt -n x}}%
\hfil\\
Set number of simulated datasets. Def: 1

\end{description}

For example, to generate two data sets using the input files in the
toy directory, and with {\tt beta=.3}, type:
\begin{verbatim}
  abcsim toy.bdf home.cfg kill.cfg -n 2 -b 0.3
\end{verbatim}

This generates two random data sets, which are written to output in
.cnt format.  Here is a sample set of output:
\begin{verbatim}
    2 #number of parts
    2 #number of data sets
  #label              DS0  DS1
  Skull                37   49
  Femur                88   84
\end{verbatim}

The simulations are done by generating {\tt alpha[i]*K} animals from
agent {\tt i}.  Each animal is generated by choosing a configuration with
the probabilities given in the relevant .cfg file, and then adding the
relevant skeletal parts to the data set.  Then, each skeletal part is
exposed to attrition.  A skeletal part of type {\tt j} survives attrition
with independent probability
\begin{verbatim}
     exp(-beta*s[j])
\end{verbatim}

where {\tt s[i]} is the sensitivity of the jth skeletal part and is
proportional to the reciprocal of the density of this skeletal part,
as given in the .bdf file.  The constant of proportionality is
adjusted so that on average, half the bones in a complete skeleton
would survive when {\tt beta=1}.  
