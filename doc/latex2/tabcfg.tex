% LaTeX document produced by pod2latex from "tabcfg.pod".
% The followings need be defined in the preamble of this document:
%\def\C++{{\rm C\kern-.05em\raise.3ex\hbox{\footnotesize ++}}}
%\def\underscore{\leavevmode\kern.04em\vbox{\hrule width 0.4em height 0.3pt}}
%\setlength{\parindent}{0pt}

\section{TABCFG}%
\index{TABCFG}

\subsection*{TABCFG Documentation File}%
\index{TABCFG Documentation File}

Tabcfg is a program that tabulates configurations.  It reads a .cfg
file and its output is in the form of a .cfg file (to find about .cfg
files, see {\em files\/}).  It examines the configurations in its input
and eliminates duplicates.  For example, consider the following 
file, named {\em small.cfg\/}:
\begin{verbatim}
  ############## small.cfg #######################################
  2   #number of parts
  5   #number of configurations
  #
  #probabilities of configurations:
                10  4  3   2   1
  #
  #Configurations:
  #label                
  Skull         0   0  1   1   1
  Femur         1   1  2   2   0
\end{verbatim}

The 1st and 2nd configurations are identical as are the 2nd and 3rd.
Running this through tabcfg produces the following output, which is
also in the form of a .cfg file:
\begin{verbatim}
  #                                   TABCFG
  #                          Tabulate Configurations
  #                             by Alan R. Rogers
  #                                Version 0.10
  #                         Type `tabcfg -- ' for help
  
  #Cmd line: tabcfg small.cfg
  #Configured agent from file small.cfg
  #Input file: small.cfg
       2  # number of parts
       3  # number of configurations
  #Probabilities of configurations are proportional to:
                     14 5 1
  #label
  Skull              0  1  1
  Femur              1  2  0
  
The new file contains the same configurations as the old one, but
now each configuration is unique.  The probability of the i'th
configuration in the new file is the sum of the probabilities
corresponding to that configuration in the old file.   The new .cfg
file is smaller and easier to manipulate than the old one.  Using the
new, shorter, .cfg file as input to abcml or abcsim will generate
precisely the same output, but with some savings in computer time.
\end{verbatim}

