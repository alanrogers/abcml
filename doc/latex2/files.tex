% LaTeX document produced by pod2latex from "files.pod".
% The followings need be defined in the preamble of this document:
%\def\C++{{\rm C\kern-.05em\raise.3ex\hbox{\footnotesize ++}}}
%\def\underscore{\leavevmode\kern.04em\vbox{\hrule width 0.4em height 0.3pt}}
%\setlength{\parindent}{0pt}

\chapter{Input Files\label{ch.files}}%
\index{Input Files}

This chapter describes the various kinds of input file that are used by the
programs in this package.  The type of a file is specified by its suffix:
\begin{verbatim}
  Suffix           File Type
  ------           ---------
   .bdf            Bone definition file
   .cfg            Agent configuration file
   .mau            Minimum Animal Units file
   .cnt            Skeletal part counts
   .wgt            Skeletal part weights
\end{verbatim}
All of these formats are plain ascii.  Comments may be included in any
input file.  A comment begins with the character \# and ends at the end
of the line.  Comments are ignored by the input routines.  None of the
programs in the package use all of these types of input file, but
several of them use more than one.

\section{Bone Definition File (.bdf)\label{sec.bdf}}%
\index{Bone Definition File (.bdf)}

The bone definition file is used to describe the characteristics of
skeletal parts.  Example:
\begin{verbatim}
    #File toy.bdf
    2   #number of parts
    #label      live    density
    Skull       1       0.49
    Femur       2       0.37
\end{verbatim}

The file toy.bdf defines the skeletal parts of a toy model in which
there are only two elements: skull and femur.  The first line of input
says that there are 2 parts.  The next line is a comment that is
ignored by the input routine.  After that, each line has three fields,
which are separated by white space (blanks and/or tabs):

\begin{description}

\item[Field 1]
the label of the skeletal part

\item[Field 2]
the number of this part in a complete skeleton

\item[Field 3] the density of this part.  The program assumes that a copy of
  skeletal part $i$ survives attrition with probability $\exp[-\beta A/d_i]$,
  where $d_i$ is the density of the skeletal part, $A$ is a scaling constant
  determined by the program, and $\beta$ measures the intensity of attrition.
  $A$ is determined by the program so that when $\beta=1$, half the bones in
  an entire skeleton would survive attrition.  Because of this rescaling, it
  doesn't matter what units density is measured in.

\end{description}

\section{Agent Configuration File (.cfg)\label{sec.cfg}}%
\index{Agent Configuration File (.cfg)}

The agent configuration file describes the characteristics of agents
of deposition.  Example:
\begin{verbatim}
    # file:  home.cfg
    2  #number of parts
    5  #number of configurations
    #probabilities of configurations:
             0.5   0.2   0.15    0.1    0.05
    #Configurations:
    #label              
    Skull    0     0     1       1      1
    Femur    1     2     2       1      0
\end{verbatim}

An animal need not be deposited in the assemblage as an entire skeleton.  It
may be deposited as a skull only, as a skull and 1 femur, as 2 femurs and no
skull, and so on.  Each of these possibilities is called a ``configuration''.
To describe an agent of deposition, the .cfg file specifies the configurations
that are possible along with their probabilities.  The lines are interpreted
as follows:

\begin{description}
\item[Line 1]
the number of skeletal parts

\item[Line 2]
the number of configurations

\item[Line 3]
numbers, separated by white space, giving the probabilities of the
configurations.  Although I refer to these as probabilities, they need
not sum to 1, for they are normalized by the input routine.  For
example, the probabilities line for a data set with 5 configurations
might look like 
\begin{verbatim}
    1 1 2 1 1
\end{verbatim}
which would imply that the probability of configuration 3 is twice as
large as those of configurations 1, 2, 4, and 5.

\item[Each succeeding line]
has several fields, separated by whitespace.  The first field is the
label for a particular skeletal part.  The remaining fields give the
number of that part in configuration~1, configuration~2, and so on.
\end{description}

For example, home.cfg tells us that there are 2 parts and 5
configurations.  The 1st configuration occurs with probability 0.5 and
consists of 0 skulls and 1 femur.

\section{Minimum Animal Units File (.mau)\label{sec.mau}}%
\index{Minimum Animal Units File (.mau)}

In published literature, one often finds skeletal part counts
expressed as Minimum Animal Units (MAUs).  These are equal to the
number of distinct copies of each part divided by the number of copies
of that part in a living animal.  The .mau file format provides the
same information as the .cfg file format except that skeletal parts
are expressed as MAUs rather than as raw counts.  The format is
identical to the .cfg format except that configurations are columns of
floating-point numbers rather than columns of integers.  Use the
program mau2cfg to convert from .mau format to .cfg format.

\section{Skeletal Part Count File (.cnt)\label{sec.cnt}}%
\index{Skeletal Part Count File (.cnt)}

The skeletal part count file is read by several of the programs in the
package.  In addition, abcsim produces output is in the form of a .cnt file so
that it can be used as input by other programs.  This file provides the counts
of skeletal parts from one or more assemblages.  Here is an example:
\begin{verbatim}
    # file: toy.cnt
    2 #number of parts
    2 #number of data sets
    #label              DS0  DS1
    Skull               537  549
    Femur               942  983
\end{verbatim}
The first two lines give the number of skeletal parts and the number
of data sets.  Each succeeding line has several fields separated by
whitespace, which are interpreted as follows:

\begin{description}
\item[Field 1]
skeletal part label

\item[Field 2]
count of this part in the 1st data set

\item[Field n]
count of this part in the (n-1)th data set
\end{description}

\section{Weight File (.wgt)\label{sec.wgt}}%
\index{Weight File (.wgt)}

The weight file format is read only by the wgtcfg program.  It allows wgtcfg
to generate a .cfg file that is consistent with the assumption that
configurations that are high in energetic value but easy to carry are most
likely to be transported.  Its format looks like this:
\begin{verbatim}
  24 # number of parts
  #part               MUI    GrossWgt
  half-mandible       295.00   889.00 # without tongue; divisor=2
  atlas/axis          262.00   315.00 # divisor = 2
  cervical_vert       272.14   301.71 # divisor = 7
  thoracic_vert       187.15   214.62 # divisor = 13
  lumbar_vert         284.33   323.33 # divisor = 6
  innominate          843.67  1058.88 # = pelvis/3
  sacrum              843.67  1058.88 # = pelvis/3
  rib                 101.92   141.81 # divisor = 26
  scapula            2295.00  2398.00 # divisor = 1
  P_humerus           743.00   830.50 # divisor = 2
  D_humerus           743.00   830.50 # divisor = 2
  P_radius            377.50   459.50 # divisor = 2
  D_radius            377.50   459.50 # divisor = 2
  carpal               33.50    46.75 # divisor = 8
  P_metacarpal         33.50    46.75 # divisor = 8
  D_metacarpal         33.50    46.75 # divisor = 8
  phalange             15.67    24.67 # divisor = 6
  P_femur            2569.50  2671.00 # divisor = 2
  D_femur            2569.50  2671.00 # divisor = 2
  P_tibia             218.33   255.33 # divisor = 6
  D_tibia             218.33   255.33 # divisor = 6
  tarsal              218.33   255.33 # divisor = 6
  P_metatarsal        290.50   377.00 # divisor = 2
  D_metatarsal        290.50   377.00 # divisor = 2
\end{verbatim}

As usual, the first column gives the labels of the skeletal parts.
The second column gives some measure of the utility of the part.  In
this example I have used the ``meat utility index'' (MUI) of Metcalfe
and Jones (1988, p.~489).  It is simply the weight (in grams, as I
recall) of the meat attached to the bone.  The third column gives the
gross weight (again, I think, in grams) of the skeletal part.

There is one minor complication.  Published data provide weights not
for individual skeletal parts but for various packages that are
thought to be transported together (Metcalfe and Jones 1988, p 489;
Binford 1978, pp 16--17).  For example, the values published for ``ribs''
refer to the entire rib cage, but those for ``femur'' refer to a single
femur (not the two femurs together).  Both the MUI and the GrossWgt
values must be divided into components that reflect the skeletal parts
used in the analysis.  If the ribs always travel together, then it
doesn't matter whether we assign the entire weight to a single rib and
assign zero weight to the other ribs, or whether we assign equal
weight to each rib.  The latter choice is simpler, however, so that is
what I have done here.  Since there are 26 ribs, I divided the
published ``rib'' values by 26.  Since the distal and proximal ends of
the femur are tabulated separately, I divided the published ``femur''
values by 2.  These divisors are given in comments at the end of each
input line to help me remember what I did.

%%% Local Variables: 
%%% mode: latex
%%% TeX-master: t
%%% End: 
